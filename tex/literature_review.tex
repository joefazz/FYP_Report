% --------------------------------------------------
% Literature Review
% --------------------------------------------------
% The literature review is an essential component of your project report. You should discuss the existing literature that is relevant to your project with full and proper referencing. You should aim to refer to a range of material including academic papers, text books, articles and existing product descriptions. It should be clear to the reader why the literature you identify is relevant and how you have incorporated the learnings from your review into your project. For example, you may have made a number of project 
% Page 4 of 5 decisions based on your review of the literature and these decisions should be described. The literature review should also lead to the creation of a number of possible solutions to your problem articulation and technical specification

\chapter{Literature Review} \label{lit}

This chapter focuses on an analysis of existing systems in the online development space and the literature that provides context to the decisions that were made during development. It looks at the advancement of the web browser and discusses the functionality provided which can lead to this type of project to be developed. It also looks at the current state of the art of virtualization as it relates to personalised environments and how the advent of Containers has fundamentally changed the way users interact with virtual environments, as illustrated by supporting literature.

\section{The Web Browser} \label{lit-web}

The web browser has been through a myriad of changes since it's birth in the early 1990s. An excellent article titled "Mosaic and the World-Wide-Web"\cite{mosaic} illustrates the problems that were prominent in the early stages of the industry such as the lack of a search engine leading to difficulty in finding resources. It also illustrates what were considered leaps in progress at the time such as Mosaic being the first browser to support in-line multimedia and to have a 'back' and 'forward' button.

Such concepts that were developed during the time are still very relevant, the general TCP/IP stack had been determined and the HTTP protocol was in use. An issue with HTTP is the protocol by design has latency embedded as it is designed for sending structured messages. For a project attempting to create an environment that mirrors a local set up online latency is a big hurdle and the need for it to be real-time is key. 

\section{Real-Time Communication} \label{lit-web}

An experiment was done in 2012 discussing the performance of different RTC methods by Professors at the University of New Brunswick\cite{websocket}.

\textbf{HTTP polling} is an attempt to solve the real-time issue however it is still built on top of a system not designed for real-time, full duplex communication. \textbf{HTTP long-polling} is another solution that sticks to the HTTP protocol but reduces the number of wasteful requests by having the server intelligently not respond to the request if there is no information available and hang until a timeout or information becomes available.

A modern solution to this is the \textbf{WebSocket} protocol proposed in RFC 6455 \cite{wsrfc} which aims to reduce latency by a factor of 3 compared to HTTP in the real-time communication aspect. It is a fully duplexed, bidirectional communication channel that uses physical sockets to connect machines.

\section{Online Developer Environments} \label{lit-ode}

A number of existing solutions providing online development environments exist and have been analysed for the purpose of this review.

\subsection{Repl.it}
Repl.it is very similar to the idea proposed in the Problem Statement (Section \ref{section:probart-probstate}) and a lot of the requirements lined out in Section \ref{section:probart-techspec}. It offers a huge array of Repl templates available for users to get started with many languages/frameworks very quickly. It also uses the Monaco Editor provided by Microsoft in order to provide a first class text editor experience.

Repl.it takes advantage of containerisation in order to gives users the full developer experience when visiting the system \cite{replit-containers}. The system also uses it's own container orchestration software in order to scale the instances available to users up and down depending on demand.

For long running processes such as a web server. Repl.it will host the process on a subdomain of the repl.it domain which is accessible at any time.

\subsection{Codecademy}
Codecademy is a educational focused online environment designed to teach users how to code. Ranging in topics from beginning web development to a course to the IBM Watson API. It is a more directed experience than Repl.it as users are performing tasks for exercises but they are typing code into a similar environment, the code is executed and the result is displayed to the user.



\subsection{Glitch}

\section{Containerisation and Virtualisation} \label{lit-containers}

https://ieeexplore.ieee.org/abstract/document/6903537

^^ Talking about containers vs VMs for PaaS software

https://ieeexplore.ieee.org/abstract/document/7158965

^^ More focused on Docker

\pagebreak
